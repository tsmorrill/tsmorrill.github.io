
%there are four versions, one for each tute and then one suplementary. 
%the ifthen package is used to hide the different versions, and to include 
%or hide solutions. 

\documentclass[12pt]{amsart}
\usepackage{a4wide,bm,graphicx,caption,subcaption}
\captionsetup[subfigure]{labelfont=rm}
\usepackage{enumitem,ifthen}

%\newcommand{\quizVersion}{4} %set to 1, 2, 3, or 4 depending on the tutorial time

\usepackage{comment}
%\specialcomment{solution}{\textbf{Solution}\quad}{}
%\newboolean{iflecturer}
%\setboolean{iflecturer}{false}
%
%\ifthenelse{\boolean{iflecturer}}{}{\excludecomment{solution}}

\pagenumbering{gobble}

\begin{document}
%-----------------------title section
\begin{center}
{\Large Introduction Survey}\\
\vspace{1cm}
Don't worry, it's not for a grade.
\vspace{1cm}
%You must show all your work to earn full marks.
%This can include algebraic formulas, numerical/graphical techniques, and named theorems from lectures.
%If you use a calculuator for a decimal approximation, include the formula you are evaluating.
%
%\vspace{1cm}

\begin{tabular}{@{}ll}
Time estimate:   &        {\rm \bf TEN} ($10$) minutes\\
No. of questions: &  {\rm\bf SIX} ($6$)\\
Total marks: & {\rm \bf ZERO} ($0$)\\
There are: & {\rm \bf ONE} ($2$) pages\\
%Table of useful integrals & Page $2$ .
\end{tabular}

\vspace{1cm}

\end{center}
%\vspace{1cm}
%Name:\underline{\hspace{6cm}} \hspace{1cm} zID:\underline{\hspace{6cm}}
\vspace{0.5cm}
%-----------------------title section


%------------------------------------------------
%      The Questions 
%------------------------------------------------

\begin{enumerate}

\item What is your full name? (The way the registrar writes it.)

\vspace{1cm}


\item What is your personal name? (What should I call you in class?)

\vspace{1cm}

\item What are your pronouns? (If no one has ever asked you this, most men use he/him,
and most women use she/her.)

\vspace{1cm}

\item How was your summer?

\vspace{5cm}

\item Describe your feelings about mathematics in a few sentences and/or one drawing of a face.

\newpage

\item What do you want to get out of this class?

\end{enumerate}
\end{document}
