\documentclass[12pt]{article}

\pagestyle{empty}
%\setlength{\topmargin}{0in}
%\setlength{\headheight}{0in}
%\setlength{\headsep}{0in}
\usepackage[hmargin=2cm,vmargin=2cm]{geometry}


\usepackage{amssymb}
\usepackage{latexsym}
\usepackage{amsmath,amsthm}
\usepackage{enumerate}
\usepackage{multicol}
\usepackage{hyperref}
\usepackage{fancyhdr}
\usepackage{graphicx}
\usepackage{enumitem}
\usepackage{pgfplots}

\pagestyle{fancy}
%\chead{\Large Partial Fraction Decomposition}
\lhead{Sex \& Gender Teach-In}
\rhead{19th March 2025}
%\lfoot{\footnotesize{Adapted from \textit{Thomas' Calculus:Early Transcendentals}}}
\cfoot{\hspace{.5cm}\thepage}
%\rfoot{\copyright Tracy Weyand}
\renewcommand{\headrulewidth}{0pt}
\renewcommand{\footrulewidth}{1pt}

\setlength\parindent{0pt}

\pagenumbering{gobble}

\begin{document}

\begin{center}
{\Large \textsc{Common Pitfalls About Pronouns}}
\end{center}

\begin{itemize}

\item \textbf{Don't argue about grammar} - The English word \emph{they} has been used as a singular pronoun since at least the writings of Chaucer in the 13th century, back when the pronoun \emph{you} was only used as a plural pronoun. Quakers in particular continued to use \emph{thee} instead of singular \emph{you} through the early 20th century as part of their practice of plain speech.

\item \textbf{Don't gawk at neopronouns} - It takes a lot of trust to tell someone that you use a pronoun besides \emph{he}, \emph{she}, or \emph{they}. Respect that trust!

\item \textbf{Pronouns are for cisgender people too} - If you think that your pronouns should be obvious because of the way you look, well\ldots

\item \textbf{Any pronoun can describe any body} - The way that you should address a person is not determined by what they look like, and there really isn't any way to tell who is transgender by sight. However\ldots

\item \textbf{Asking for pronouns can be tricky} - Depending on how well you know someone, and the inflection of your voice, asking for someone's pronouns can carry a nasty subtext like ``so exactly what kind of freak are you?'' It's best to let a person to disclose their pronouns when they feel safe. So\ldots

\item \textbf{Don't berate anyone for not disclosing their pronouns} - There are lots of reasons that a person might not be comfortable disclosing their pronouns. Maybe they're still figuring things out. Maybe they don't feel safe outing themself in the present moment. What you \emph{can} do is\ldots

\item \textbf{If someone is out, use their pronouns even when they aren't around} - Misgendering someone while they are absent is a social cue that you think they are an acceptable target of humiliation and possibly violence. It is disrespectful. On the other hand\ldots

\item \textbf{If someone is closeted, use the pronouns that keep them safe} - You might be asked by a friend to use different pronouns for them around, for example, their parents. If this happens, follow their lead and \emph{do not} try to make a statement on their behalf. More broadly\ldots

\item \textbf{Don't pick a fight when it's not your neck on the line} - If you want to stand up for a friend or a colleague when they are misgendered, that's great! But first, you need to be absolutely	sure that's what your friend wants. It's easy to unintentionally stir up even more trouble that your friend will have to deal with on their own. On the flip side\ldots

\item \textbf{When you make a mistake, apologize, but don't grovel} - It's tiring to hear someone self-flagellate over a faux pas. This kind of behavior falsely centers you as the injured party when you are not. Honestly, mistakes happen a lot and you're not that special! And especially\ldots

\item \textbf{Don't argue back to someone about their pronouns, EVER!} - It is sadly common for people to weaponize the pronoun \emph{they} to avoid, for example, addressing a transgender woman as she. This isn't a win for gender neutral language, it's transmisogyny.

\end{itemize}

\vfill

\end{document} 